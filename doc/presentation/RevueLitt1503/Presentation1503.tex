\documentclass{beamer}
\usepackage[frenchb]{babel}
\usepackage[utf8]{inputenc}
\usepackage[T1]{fontenc}
\usetheme{Goettingen}

\begin{document}


\title[identification DMRs] % (optional, only for long titles)
{Analyse différentielle des régions methylées}
\subtitle{Revue de littérature}
\author{Jakobi Milan}
\institute{TIMC-Imag}
\date[KPT 2004]{Vendredi 15 mars 2019}
\titlegraphic{\includegraphics[width=0.5\textwidth,height=2cm]{logo_timc.png}}






\frame{\titlepage}

\begin{frame}
\frametitle{Table des matières}
\tableofcontents
\end{frame}


\section{DMRcate}
\subsection{Données}

\begin{frame}
\frametitle{Données}
\begin{itemize}
\item<1-> 450k microarray avec 25\% des probes sur des zones intercalaires.
\item<1-> On utilise les M-values ( $log(\beta)$)
\item<2-> Méthode extensible à toutes données génomique ( RRBS, WGBS...).

\end{itemize}
\end{frame}

\subsection{Méthode}
\begin{frame}
\frametitle{Présentation de la méthode}
3 types d'analyse proposées :
\begin{enumerate}
\item<2-> Analyse entre deux groupes ( Traitement vs Contrôle)
\item<2-> Analyse de contraste.
\item<2-> Analyse de variabilité ( On identifie alors les VMRs).
\end{enumerate}
\end{frame}

\begin{frame}
\frametitle{Pseudo-algorithme}
Après avoir choisi le type d'analyse, les étapes de la procédure seront les suivantes :
\begin{enumerate}
\item<2-> On calcule $Y_{i}$ nos statistiques de test.
\item<3-> On estime la distribution de nos statistiques de test $Y_{i}$ par noyau gaussien ( par zones de taille $\lambda$).
\item<4-> On modélise, par méthode de Satterthwaite, nos statistiques de test "smoothées".
\item<5-> On en déduit les pvaleurs
\item<6-> On fixe un seuil à partir duquel on exclue les variables dont la pvaleur est trop forte
\item<7-> On construit nos DMRs/ZMRs finales en regroupant les CpG sites qui sont au plus à $\lambda$ nucléotides
\end{enumerate}
\end{frame}










\section{blabla}

\begin{frame}
\end{frame}
\end{document}