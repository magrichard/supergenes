\documentclass{beamer}
\usepackage[frenchb]{babel}
\usepackage[utf8]{inputenc}
\usepackage[T1]{fontenc}
\usetheme{Goettingen}

\begin{document}


\title[identification DMRs] % (optional, only for long titles)
{Analyse différentielle des régions methylées}
\subtitle{Revue de littérature}
\author{Jakobi Milan}
\institute{TIMC-Imag}
\date[KPT 2004]{Vendredi 15 mars 2019}
\titlegraphic{\includegraphics[width=0.5\textwidth,height=2cm]{logo_timc.png}}






\frame{\titlepage}

\begin{frame}
\frametitle{Table des matières}
\tableofcontents
\end{frame}


\section{DMRcate}
\subsection{Données}

\begin{frame}
\frametitle{Données}
\begin{itemize}
\item<1-> 450k microarray avec 25\% des probes sur des zones intercalaires.
\item<1-> On utilise les M-values ( = $log(\beta)$ )
\item<2-> Méthode extensible à toutes données génomiques : RRBS, WGBS...

\end{itemize}
\end{frame}

\subsection{Méthode}
\begin{frame}
\frametitle{Présentation de la méthode}
3 types d'analyse proposées :
\begin{enumerate}
\item<2-> Analyse entre deux groupes ( Traitement vs Contrôle)
\item<2-> Analyse de contraste.
\item<2-> Analyse de variabilité ( On identifie alors les VMRs).
\end{enumerate}
\end{frame}

\begin{frame}
\frametitle{Pseudo-algorithme}
Après avoir choisi le type d'analyse, les étapes de la procédure seront les suivantes :
\begin{enumerate}
\item<2-> On calcule $Y_{i}$ nos statistiques de test.
\item<3-> On estime la distribution de nos statistiques de test $Y_{i}$ par noyau gaussien ( par zones de taille $\lambda$).
\item<4-> On modélise, par méthode de Satterthwaite, nos statistiques de test "smoothées".
\item<5-> On calcule les pvaleurs du modèle.
\item<6-> On fixe un seuil à partir duquel on exclue les variables dont la pvaleur est trop forte
\item<7-> On construit nos DMRs/ZMRs finales en regroupant les CpG sites qui sont au plus à $\lambda$ nucléotides
\end{enumerate}
\end{frame}

\begin{frame}
\frametitle{Construction des $Y_i$}
Selon le type d'analyse, nos statistiques de test $Y_i$ sont différentes :
Pour les analyses entre deux groupes ou analyse de contraste, on a :
$$Y_i = \hat{t}^2$$
avec $t^2$ la statistique du test de Fisher modéré ( ratio de la M-value sur son écart-type).
Pour l'analyse de variabilité, on a :
$$Y_i = \frac{V_i}{V}$$
avec $V_i$ la variance des M-values de l'échantillon $i$ et $V$ la moyenne de cette variance sur tous les échantillons. ( Asymptotiquement équivalent à $F_{n-1,\infty}$).
\end{frame}

\begin{frame}

\frametitle{Kernel smoothing}
Pour l'estimation par noyau,  chaque noyau est construit sur une longueur de $\lambda$ nucléotides, et le paramètre d'échelle $\sigma$ est déterminé par la relation suivante :
$$\sigma = \frac{\lambda}{C}$$
Avec $C$ l'unique hyperparamètre de la méthode ( $C^* $ déterminé par CV). On pose :

\begin{equation}
\left\lbrace
\begin{array}{ccc}
S_{KY}(i) = \sum^n_{j=1}K_{ij}Y_{j} \\
S_{K}(i)= \sum^n_{j=1}K_{ij} \\
S_{KK} = \sum^n_{j=1}K_{ij}^2
\end{array}\right.
\end{equation}



\end{frame}


\begin{frame}{Sélection des CpG sites}

On pose (Satterwhaite) :
\begin{equation}
\left\lbrace
\begin{array}{ccc}
a_{i} = \frac{S_{KK}(i)}{ (\mu S_{K}(i)} \\
b_{i} = \frac{\mu S^2_{K}}{S_{KK}(i)}
\end{array}\right.
\end{equation}
et on teste $\frac{S_{KY}(i)}{a_{i}} \sim \chi^2_{b(i)}$ \\
Les pvaleurs de ce test sont corrigées par procédure Bonjemini-Hochberg.  On retient les sondes dont la pvaleur est inférieure au seuil choisi ( l'auteur conseille 0.05). \\
On finit par construire nos DMRs ( ou VMRs) en regroupant les CpG sites qui sont à au plus $\lambda$ nucléotides de distance.
\end{frame}


\subsection{Bilan}
\begin{frame}
\frametitle{Pros et Cons}


\begin{itemize}
\color{green}
\item Peu d'hyperparamètres : permet de s'affranchir des artefacts du jeu de données
\item 95 \% de recall sur jeu de données simulés.
\item Les auteurs ont pu retrouvé plusieurs ZMRs qui revenaient à chaque fois selon le type de tissu utilisé.
\item En comparant à d'autres méthodes utilissant aussi $limma$, les résultats étaient en moyenne meilleure et au moins aussi rapide.
\item A l'étape de construction des zones ( après sélection des CpG sites, pourquoi ne pas faire du ML non supervisé?)
\item ?

\color{red}
\item coût calculatoire élevé ( estimation par noyau).
\item Taille des DMRs non contrôlable puisqu'adossée au paramètre d'échelle d'estimation par noyau + tailles virtuellement similaires ( max $2\lambda$.
\item ?
\end{itemize}
\end{frame}






\end{document}